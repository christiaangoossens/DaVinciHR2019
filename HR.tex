\documentclass[a4paper]{article}
\usepackage{Style/DaVinci2019}
\usepackage{enumitem} % http://ctan.org/pkg/enumitem
\usepackage[none]{hyphenat}

\newcommand{\Abr}{Administrative Regulations} % Bestuursregelement
\newcommand{\Asta}{Bylaws} % Statuten
\newcommand{\Ahr}{House Rules} % Huishoudelijk Regelement
\newcommand{\Asr}{Safety Rules} % Veiligheidsregelement
\newcommand{\Awr}{Contest Rules} % Wedstrijdregelement
\newcommand{\Ajv}{Annual Report} % Jaarverslag
\newenvironment{g}{\color{grey}}{}

\setTitle{HR Draft}
\setSubtitle{As defined by the HR Committee 2019}
\enableFooter
\linenumbers
\setFooter{Huishoudelijk Regelement}

\begin{document}
\tableofcontents
\pagebreak
\section*{List of Definitions}
{\g Definitions can be recognized by the capital letter at the start throughout this document. Board Member roles are also spelled with a capital (Chairman, Secretary, Treasurer)}

\begin{description}[font=\sffamily\bfseries, leftmargin=1cm, style=nextline]
  \item[GA]
    General Assembly or General Members Meeting (ALV)
  \item[BM]
    Board Meeting (BV)
  \item[Workshop]
    An event organized for an external party, with an aim of presenting the sport to outsiders. If desired it is possible to try out archery under guidance of experienced archers.
    \item[ESSF]
    Eindhovense Studenten Sport Federatie (Eindhoven Student Sport Federation)
    \item[HR]
    House Rules (this document)
    \item[NHB]
    Nederlandse Handboog Bond (Dutch Archery Federation)
    \item[Shooting line]
    The line, marked on the floor, from where the archer shoots his arrows towards the target.
    \item[SSC]
    Student Sports Centre Eindhoven (also SSCE)
    \item[TU/e]
    Technische Universiteit Eindhoven (Eindhoven University of Technology)
    \item[Safety Rules]
    The Safety Regulations of E.S.H. Da Vinci can be found in this HR under Chapter 1.
    \item[Association Year]
    Parallel to the financial year; starting September 1st and ending the 31st of August. 
    \item[“Behind the line”]
    The side of the line where the targets are NOT placed. 
    \item[Waiting line]
The line placed behind, and parallel to, the shooting line. Only supervisors, trainers and participants currently shooting their turn are allowed to be in the space between the shooting line and the waiting line. 
\item[Club Room] Also known as “het hok”. It’s the lockable area/room in the back of the shooting range, assigned for use by E.S.H. Da Vinci.
\item[Member]
Archers that are members of the E.S.H. Da Vinci association. Members are required to have a sport card from the SSC.
\item[External Member]
Archers that are members of the E.S.H. Da Vinci association, but are NHB member through another club.
\item[Recreationist]
Archers that are allowed to take part in trainings. Recreationists are required to have a sport card from the SSC.
\item[Beginner]
Participants to the Beginners' Course
\item[Beginners' Course] Course given to beginners during the year.
\item[Guest] Guests of Da Vinci, this includes beginners, workshop participants or interested parties.
\item[General Training] Training given by the trainers during a defined time.
\item[Honorary Member] Special title given to an archer by the GA for services to Da Vinci.
\item[Experienced - Inexperienced] The board determines if archers are experienced, or inexperienced depending on their safety level and skills.
\item[Supervisor] TO BE DEFINED
\item[Downrange] TO BE DEFINED
\item[Accomodation] Our accommodation ("Aan de Meet") consisting of the Shooting Range and the Club Room.
\end{description}

\section{Safety Rules}
The shooting range is a safe place, provided that safety rules described here are followed. Every archer should know and observe these rules. We also expect archers to ensure others abide by these rules. \\

Archers violating these rules may be removed from the shooting range. \\



{\bf A. General}

\begin{enumerate}
  \item Archers must at all times use common sense and observe the safety of everyone.
  \item All archers must immediately follow safety related instructions from supervisors. The correctness of such instructions is not open for discussion until the instructions have been complied with.
  \item Inexperienced archers are only allowed to shoot under supervision, and should strictly follow all instructions from their supervisors. Experienced archers are allowed to shoot unsupervised.
  \item Aiming a bow at persons or animals is strictly forbidden (even without an arrow on the bow).
  \item When drawing and/or shooting the bow, the angle of the arrow with the ground (elevation) may never exceed 30$^\circ$. Shooting straight up is strictly forbidden.
  \item A drawn bow without an arrow on it may never be released. Doing so may cause the bow to shatter, causing injury and/or damage.
\end{enumerate}



{\bf B. Shooting protocol}

\begin{enumerate}
  \item Bows may only be drawn at the shooting line, and only when aimed in the direction of the targets. This also holds when there's no arrow on the bow.
  \item Arrows may only be placed on the bow when at the shooting line. The arrow tip must always be pointed in the general direction of the targets, or the ground.
  \item Archers may only take place at the shooting line when there is no person downrange of the shooting line {\bf and} (for the indoor range) the back door is closed. Until that time, they must wait behind the waiting line.
  \item During shooting only active archers and supervisors may be at the shooting line. Everyone else, including archers who finished shooting, must wait behind the waiting line.
  \item If, during shooting, anyone is found to be downrange of the shooting line, or (for the indoor range) the backdoor is found to be open, all archers will immediately stop shooting. Shots are aborted, not finished, and the arrows must be removed from the bows.
  \item When all archers have finished shooting, everyone may fetch their arrows collectively.
\end{enumerate}



{\bf C. Ranges}

\begin{enumerate}
  \item There may be no objects (such as targets or scoring boards) on the floor within 3 meters of the target wall.
  \item There will be no running, jumping, pushing, etc. on the range. Reckless behaviour is prohibited.
  \item Bows should be put in the designated areas, or in the bow rack.
  \item Spectators are not allowed downrange of the waiting line, unless explicitly permitted by supervisors.
\end{enumerate}



{\bf D. Outdoor range}

\begin{enumerate}
  \item Archers are required to wear closed shoes on the outdoor range to protect against obscured arrows stuck in the grass.
  \item Archers who are shooting on the outdoor range must pay extra attention that all archers have returned after fetching their arrows, including those shooting at the 70 or 90 meter targets.
\end{enumerate}

\section{General}
\subsection{Goal and Scope of the \Ahr}
These \Ahr\ supplement the \Asta\ with regards to rules and conditions contributing to the day-to-day functioning of the association. Members (including the board) and Recreationists are tasked with knowing the contents of this document and following the described rules, which will be surveyed by the board. \\

\subsection{Behaviour}
The NHB provides additional regulations regarding behavior and rules at competitions, which have to be followed by all archers at NHB competitions. Further regulations about safe behavior on the shooting range, as well as general competition rules, are part of the \Awr\ and the \Asr . Archers are always required to behave in accordance to the \Asr , which the board and trainers will survey. \\

Members and Recreationists are obliged to try and keep the association going as best they can and are obliged to behave appropriately and to not embarrass Da Vinci (as long as they are in direct association with Da Vinci). \\

While not mandatory, assistance at and supporting activities organized by the association is appreciated.

\subsection{Clothing Rules}
\label{section:clubclothing}
At regional and team competitions, organized by the NHB, (non-External) Members are required to wear a club shirt, together with a black pair of pants without any markings. This rule also applies if one is representing Da Vinci, for instance during the intro, the Beginners' Course or at another competition where one is using the Da Vinci name, or present with a larger team from Da Vinci. At NHB competitions, the NHB regulations for clothing should be followed as well (and thus black jeans are not allowed).

\subsection{Incorporation of the Association}
The association is incorporated at the Kamer van Koophandel under number 17174167. The association is also a member of the ESSF, based at the SSC, and the NHB under number 1364.

\subsection{Changes to the \Ahr}
Proposals to change the \Ahr\ shall be submitted at least 14 days in advance of the next GA by at least three Members, or the board. An overview of these changes will be communicated by written notice by the board at least 7 days in advance of the GA. Proposed changes to the \Ahr\ will only go into effect after they have been approved by the GA. Changes in the \Asr\ can be implemented directly by the board, without GA approval, as they improve safety. Changes to the \Asr\ have to be discussed in the next GA.

\subsection{Validity of the \Ahr}
The \Ahr\ will be declared invalid upon dissolvement of the association, or when the \Asta\ are changed. Upon changing of the \Asta , the \Ahr\ have to be re-approved to make sure that they are not alien to the \Asta . The \Ahr\ should be checked at least once a year for inconsistencies with day-to-day rules, and updated if necessary. New rules decided upon by the GA should also be added to the \Ahr .

\subsection{Availability of the \Ahr}
At least one printout of the \Ahr\ and \Asta\ should always be present at the training location. Every Member and Recreationist is entitled to review these regulations, to make sure that nobody can claim unfamiliarity with these rules. The \Ahr\ and \Asta\ should also be available on the E.S.H. Da Vinci website.

\subsection{Fines and Suspensions}
Whenever an Archer infringes upon one or more articles from the \Asta , \Ahr\ or \Asr\, or in case of infringing upon regulations based on the rules described in these documents, the board can punish the violator with a fine and/or suspension. The severity of this measure will be determined depending on the specific infringement. In case of disagreement between the board and the violator, the matter will be mediated by a GA.

\subsection{General Assembly}
\subsubsection{Organization}
The board is obliged to, as stated in the statutes, organize a GA at least once every year.  The board shall provide an agenda and invitation and send these together with the announcement of the date of the GA. The GA is convened by written notice to the persons entitled to vote within a period of at least seven days.  

\subsubsection{Voting}
The voting or poll takes place as a rule by show of hands and only in person. If the board, or at least one Member or Recreationist deems it desirable, the chairman may decide to a written vote or poll, in accordance to \Asta\ Section 12.4. If a vote or poll concerns people (for instance the choosing of a new board), it shall always be a written vote. \\ 

In the event of a written vote, the chairman designates a voting committee consisting of two members. The board provides authenticated voting papers. The numbers of voting papers handed in must be equal to the numbers of voting members present. The committee counts the votes and checks the validity of the votes. The committee will announce the result of the vote to the GA. \\

A voting ballot shall be declared invalid in case of the following: it is signed, it is not marked by the board, inreadable or if it contains a choice that was not part of the set of choices that is currently being voted on.

\section{Members and Recreationists}
\subsection{Requirements}
All Members and Recreationists are required to conform to the SSC rules for joining sports associations, including, but not limited to, the rule requiring a valid student sport card.

\subsection{Contribution}
Members and Recreationists should pay the contribution as determined yearly by the GA before the deadline indicated by the board. If a Member or Recreationist has not done so, the board is permitted to undertake the actions as described in the \Asta . \\

The contribution is as follows:
\begin{enumerate}
\item Members pay €75,- This includes the NHB contribution, which will directly be paid by Da Vinci
\item External Members pay €40,-
\item Recreationists pay €40,-
\end{enumerate}


Honorary Members only pay for the NHB membership fee through Da Vinci. Honorary Members can shoot at Da Vinci for free. \\

It is also possible to become a financial donor, as specified in the \Asta\ Section 6. The minimal donation to become a financial donor is €20,-.

\section{Board}
\subsection{Communication}
\subsubsection{Between boardmembers}
The board has to hold meetings as often as the chairman or two other board members find this necessary. The board needs to meet at least once every month.
\subsubsection{To the association}
The board needs to communicate all information that is important for members and recreationists and the association, to all members, recreationists and trainers. All general notifications need to be posted on the website. Personal messages need to be emailed to the person in question, such that they can be archived. Official documents, such as GA Agenda, Minutes, etc, need to be communicated per email. \\
 
All upcoming activities and up to date training times need to be published on the website. 
General announcements, updates about upcoming activities and urgent changes in planning will also be communicated via the main communication method, currently the Da Vinci Info App in the Whatsapp application.

\subsection{Accomodation}
The board is responsible for opening and closing of the accommodation on club evenings. Exemptions to regular opening times must be communicated via the website and other main methods of communication at least one week before the exemption. The board is allowed to close prematurely if: \\

\begin{enumerate}
\item There are no active archers on the shooting range.
\item They have notified the premature closure on a main mode of communication.
\item There are no objections within 30 minutes.
\end{enumerate}

\subsection{Election}
A new board is elected during a GA. At least three weeks before the GA, the board will notify all Members and Recreationists with a prior notification that a GA will be held where a board will be elected. Candidates must apply for a board position by written notice to the secretary of the board within one week of the prior notification. The board will make the candidates known at the time the agenda for the GA is send. Only Members may be board candidates. \\

\subsubsection{Insufficient Candidates}

If there are not enough candidates to fill all required board positions, the board will notify all Members that the application period is extended to the start of the election vote of the GA. \\

If there are enough candidates to fill all required board positions by the end of the regular application period, the board will not accept new applications. \\

If the candidates are not chosen to be board members by the GA, Members can apply for a board position during the GA. If enough candidates are found and a board is chosen it can be installed at the same GA. If no board is chosen, the current board will remain installed. \\

\subsubsection{Inability to Elect a Board}

If no new board is elected at the GA, the effective board is entitled to suspend all activities including training, free practice and all other club activities until new candidates are found. The board is obligated to organize a GA with the intent to elect and install a new board as soon as enough candidates have applied. \\

If no new board can be found within three months after original GA, the board is entitled to organize a GA with the purpose to either elect and install a new board or, if no new board can be found, initiate the termination of the association.

\section{Committees}
\subsection{Establishment}
The board can establish a committee with a clear task description by communicating its establishment at a GA or by using a written notice, sent to members and recreationists, which includes the task description. This written notice will include the initial committee members.

\subsection{Dissolution}
\subsubsection{Dissolution by the board}
The committee can be dissolved by the board or GA when they have decided that its task is accomplished or obsolete, or if all tasks have been distributed over other committees (or the board). When a committee is manually disestablished, the board will sent a written notice of dissolution to all members and recreationists.

\subsubsection{Dissolution upon task completion}
The committee can also be dissolved automatically, if this was decided at establishment. These committees will be automatically dissolved after they have accomplished their tasks, and are used for one-time events. When such a committee has completed their tasks, a written notice of dissolution will be sent to all members and recreationists by the board.

\subsubsection{Dissolution notice}
Dissolution notices for committees should contain when the committee will be dissolved and a final report about their tasks.

\subsection{Structure}
Every committee has a chairman, who is responsible for presenting results to the board and maintaining order and effective task execution within the committee. Every committee will also be appointed a board member as a board contact, who will serve as a first contact point, and will stay up to date with current committee tasks.

\subsection{Reporting}
Committees are tasked with reporting their status to the board before every GA. Committees also have to report their status to the board when asked. Committees can present regarding their committee at the GA, provided they provide their report to the board at least 14 days before the GA, such that it can be archived.

\subsection{Committee members}
\subsubsection{Joining}
Both Members and Recreationists can join committees after approval of the committee chairman. The board or GA can object to new committee members, provided that they supply both the joining member, as well as the committee itself of a written reasoning. The committee chairman is responsible for informing the board when members has requested to join committees.

\subsubsection{Leaving}
Committee members can leave their committee after they have completed or transferred all their tasks. If tasks cannot be transferred to other committee members, the leaving member will have to complete their task first. The committee chairman is responsible for informing the board when members have requested to leave and can decide if one is allowed to leave the committee. \\ 

The board or GA can also force members to leave committees, provided they supply both the leaving member, as well as the committee itself of a written reasoning.

\subsubsection{Disputes}
In case that disputes arise between the board and committees (and its members), the GA will mediate.

\subsection{Budgets}
Committees have to pass every planned purchase by the Treasurer, who will approve or deny requests. Planned purchases totaling to more than €100,- require a Budget to be submitted to, and approved by the board. Budgets totaling more than €350,- have to be approved by the GA.

\subsection{Workgroups}
Committees may have workgroups, which can have members from outside the committee, to take care of a very specific task. These workgroups have a chairman, who is in the committee as well, and reports to the committee. Budgetting, reporting, joining and leaving go through the committee.


\section{Finances}
\subsection{General}
The board is responsible for keeping track of all income and expenses in the financial administration. They are also required to submit the Budget, Balance and Realization to the GA for approval.

\subsection{Club money box}
The club money box is the responsibility of the Treasurer and he/she must ensure that money is only borrowed to purchase goods for the club.

\subsection{Archival}
The Treasurer is also responsible for making sure that receipts and invoices are kept for all purchases, both cash and digital. The Treasurer is also responsible for keeping track of declarations, and making sure they are signed.

\subsection{Purchase of Goods}
The board can purchase goods for the club. Board members can spend a maximum of €50,- without consultation of the other board members. A board majority is necessary to approve a purchase of more than €50,-. In this case the purchase must be discussed and decided on during a BM. If a purchase exceeds €350,- the purchase needs to be approved through a GA.

\subsection{Financial Committee}
The Financial Committee is appointed by the GA to check the execution of the financial policy by the board. It checks the administration of the Treasurer to ensure correctness, traceability and adherence to the submitted Budget. The Financial Committee consists of two committee members. They will be appointed during the GA. One backup committee member will also be appointed.

\section{Club Materials}
\subsection{Definition}
Club Materials include bows, arrows, personal protection material, spare parts, (outdoor) targets, target faces, other material owned by the club and are used for archery. \\

Club Materials can be used by any person who is permitted by the board to use the materials. The board can always limit usage of Club Materials if needed. The Club Materials are primarily used during the Beginners' Course and Workshops, other usage is permitted only if this activities are not compromised.

\subsection{Maintenance and Repairs}
\subsubsection{Beginners' Course and Workshops}
During the Beginners' Course and Workshops, trainers are responsible for inspecting the material for damage at the beginning and end of each training or workshop. Damaged materials must not be used until repaired if safety is compromised. The usage of club materials outside the Beginners' Course and Workshops is regulated according to Section \ref{section:clubMaterialInCourse}.

\subsubsection{General}
The board is responsible for periodic and corrective maintenance of the Club Materials to ensure the materials are safe to use. The board can delegate maintenance to the Material Committee. 

\subsubsection{Damages}
In case of any damage during usage in General Training, free practice or external use, the archer who used the bow is responsible for contacting the board. In case the damage occurred to misuse, the archer is also responsible for repair costs. In case of normal wear and tear it is highly appreciated that archers who use the material play an active role in repairing it.

\subsection{Internal Use}
\subsubsection{Beginners}
Beginners are required to use Club Materials during the Beginners' Course.

\subsubsection{General}
After the Beginners' Course, Members and Recreationists are allowed to use the Club Materials during training and free practice. In case there are not enough bow and/or arrows available, the bows are distributed on a first come first serve basis.

\subsubsection{Costs}
During the first 6 months, using club bows and arrows is free of charge. After 6 months a small fee needs to be payed per club evening (now: 0,50 euro). The board and trainers from the SSC can always claim a bow for instruction purposes. In case someone payed a fee for that bow that evening, the fee will be restituted.

\subsection{External Use}
Members are allowed to borrow Club Material for use in external competitions. When borrowing a fee is required equal to the fee of two evenings for each period between two club evenings. If the bow is not returned before the beginning of the next club evening, a fee for that evening needs is required. A deposit of 50 euros needs to be paid, any damage needs to payed from this deposit. In case damage exceeds the amount of the deposit, the member is still required to pay for the damage. \\

In case Club Material is needed for a beginners course, the material needs to be returned in good order at the latest at the start of the lesson. In case the material is not in good order or returned too late a fine of 10 euro is to be paid. If no valid reason is provided for not returning the material in good order on time, the board can decide that the Member is no longer allowed to use Club Materials.


\section{Accomodation}
\subsection{Opening Hours}
The Accomodation is opened between

\subsection{Maintenance and Cleaning}
Members and Recreationists can be requested to help with maintenance and cleaning of the Accomodation by the board. Maintenance costs will be paid for by the association. If necessary, financial help may be asked for from the SSC.

\subsubsection{Small tasks}
Members and Recreationists are expected to perform small maintenance, such as throwing out the trash, themselves without board involvement.

\subsection{Shooting Range}


\subsection{Club Room}
\subsubsection{Usage rules}
The Club Room may be used by Members and Recreationists at Da Vinci, as well as by participants of the Beginners' Course, as well as those who have been granted access by the board. 

\subsubsection{Safety}
To prevent holdups at the shooting range, the door to the Club Room will be closed whenever nobody is present, or when archers are present for longer than a minute. The door will only be reopened after knocking, in accordance with the \Asr .

\subsubsection{Storing personal belongings}
Members and Recreationists are permitted to store their archery equipment, such as bows and arrows in the Club Room in spots destined for storage, as decided by the board. These materials are stored without any insurance policy, and are stored at own risk. The board will ensure safety of stored materials on a best effort basis. Stored items must be labeled with some indication of the owners identity. The board may also request the removal of items with a one month notice.


\section{Training}

\section{Competitions and Activities}
\subsection{General}
All internal competitions and activities are considered to be a training. Both Members and Recreationists can participate. Additional regulations for competitions are described in the \Awr .

\subsection{Internal Competitions}
The board is required to organize the following internal competitions: \\
\begin{enumerate}
\item \textbf{Ladder competition}: During every academic year a ladder competition is held, open to all Members and Recreationists.
\item \textbf{Koningsschieten}: Koningsschieten is held in the week of the founding of Da Vinci, as close as possible to the 2nd of February.
\end{enumerate}

\subsection{External Competitions}
\subsubsection{Yearly NHB competitions}
The NHB organizes two yearly competitions, bondscompetities, one on 18 meters and one on 25 meters. Each competition consist of a number of matches in the region (regiocompetitie), rayon and national. Da Vinci Members can participate in these competitions both individual as in teams. The board is responsible to notify Members that they can participate, the board is also responsible for subscribing Members and teams. In case Members participate in one of these competitions, the board will publish competition dates on the website. During the whole competition members are required to wear the club clothing, as described in Section \ref{section:clubclothing}.

\subsubsection{Other competitions}
The board should relay invitations to other competitions and publish dates and organize that a group of members participates in such a competition.
\end{document}