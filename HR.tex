\documentclass[a4paper]{article}
\usepackage{Style/DaVinci2019}
\usepackage{enumitem} % http://ctan.org/pkg/enumitem
\usepackage[none]{hyphenat}

\newcommand{\Abr}{Administrative Regulations} % Bestuursregelement
\newcommand{\Asta}{Bylaws} % Statuten
\newcommand{\Ahr}{House Rules} % Huishoudelijk Regelement
\newcommand{\Asr}{Safety Rules} % Veiligheidsregelement
\newcommand{\Awr}{Contest Rules} % Wedstrijdregelement
\newcommand{\Ajv}{Annual Report} % Jaarverslag
\newenvironment{g}{\color{grey}}{}

\setTitle{HR Draft}
\setSubtitle{As defined by the HR Committee 2019}
\enableFooter
\linenumbers
\setFooter{Huishoudelijk Regelement}

\begin{document}
\section*{List of Definitions}
{\g Definitions can be recognized by the capital letter at the start throughout this document. Board Member roles are also spelled with a capital (Chairman, Secretary, Treasurer)}

\begin{description}[font=\sffamily\bfseries, leftmargin=1cm, style=nextline]
  \item[GA]
    General Assembly or General Members Meeting (ALV)
  \item[BM]
    Board Meeting (BV)
  \item[Workshop]
    An event organized for an external party, with an aim of presenting the sport to outsiders. If desired it is possible to try out archery under guidance of experienced archers.
    \item[ESSF]
    Eindhovense Studenten Sport Federatie (Eindhoven Student Sport Federation)
    \item[HR]
    House Rules (this document)
    \item[NHB]
    Nederlandse Handboog Bond (Dutch Archery Federation)
    \item[Shooting line]
    The line, marked on the floor, from where the archer shoots his arrows towards the target.
    \item[SSC]
    Student Sports Centre Eindhoven (also SSCE)
    \item[TU/e]
    Technische Universiteit Eindhoven (Eindhoven University of Technology)
    \item[Safety Rules]
    The Safety Regulations of E.S.H. Da Vinci can be found in this HR under Chapter 1.
    \item[Association Year]
    Parallel to the financial year; starting September 1st and ending the 31st of August. 
    \item[“Behind the line”]
    The side of the line where the targets are NOT placed. 
    \item[Waiting line]
The line placed behind, and parallel to, the shooting line. Only supervisors, trainers and participants currently shooting their turn are allowed to be in the space between the shooting line and the waiting line. 
\item[Club Room] Also known as “het hok”. It’s the lockable area/room in the back of the shooting range, assigned for use by E.S.H. Da Vinci.
\item[Member]
Archers that are members of the E.S.H. Da Vinci association. Members are required to have a sport card from the SSC.
\item[External Member]
Archers that are members of the E.S.H. Da Vinci association, but are NHB member through another club.
\item[Recreationist]
Archers that are allowed to take part in trainings. Recreationists are required to have a sport card from the SSC.
\item[Beginner]
Participants to the Beginners' Course
\item[Beginners' Course] Course given to beginners during the year.
\item[Guest] Guests of Da Vinci, this includes beginners, workshop participants or interested parties.
\item[General Training] Training given by the trainers during a defined time.
\item[Honorary Member] Special title given to an archer by the GA for services to Da Vinci.
\item[Experienced - Inexperienced] The board determines if archers are experienced, or inexperienced depending on their safety level and skills.
\end{description}

\section{Safety Rules}


\section{General}
\subsection{Goal and Scope of the \Ahr}
These \Ahr\ supplement the \Asta\ with regards to rules and conditions contributing to the day-to-day functioning of the association. Members and Recreationists are tasked with knowing the contents of this document and following the described rules, which will be surveyed by the board.

\subsection{Rules regarding Safe Behavior and Competitions}
The NHB provides additional regulations regarding behavior and rules at competitions, which have to be followed by all archers at NHB competitions. Further regulations about safe behavior on the shooting range, as well as general competition rules, are part of the \Awr\ and the \Asr . Archers are always required to behave in accordance to the \Asr , which the board and trainers will survey.

\subsection{Clothing Rules}
At regional and team competitions, organized by the NHB, (non-External) Members are required to wear a club shirt, together with a black pair of pants without any markings. This rule also applies if one is representing Da Vinci, for instance during the intro, the Beginners' Course or at another competition where one is using the Da Vinci name, or present with a larger team from Da Vinci. At NHB competitions, the NHB regulations for clothing should be followed as well (and thus black jeans are not allowed).

\subsection{Incorporation of the Association}
The association is incorporated at the Kamer van Koophandel under number 17174167. The association is also a member of the ESSF, based at the SSC, and the NHB under number 1364.

\subsection{Changes to the \Ahr}
Proposals to change the \Ahr\ shall be submitted at least 14 days in advance of the next GA by at least three Members, or the board. An overview of these changes will be communicated by written notice by the board at least 7 days in advance of the GA. Proposed changes to the \Ahr\ will only go into effect after they have been approved by the GA. Changes in the \Asr\ can be implemented directly by the board, without GA approval, as they improve safety. Changes to the \Asr\ have to be discussed in the next GA.

\subsection{Validity of the \Ahr}
The \Ahr\ will be declared invalid upon dissolvement of the association, or when the \Asta\ are changed. Upon changing of the \Asta , the \Ahr\ have to be re-approved to make sure that they are not alien to the \Asta . The \Ahr\ should be checked at least once a year for inconsistencies with day-to-day rules, and updated if necessary. New rules decided upon by the GA should also be added to the \Ahr .

\subsection{Availability of the \Ahr}
At least one printout of the \Ahr\ and \Asta\ should always be present at the training location. Every Member and Recreationist is entitled to review these regulations, to make sure that nobody can claim unfamiliarity with these rules. The \Ahr\ and \Asta\ should also be available on the E.S.H. Da Vinci website.

\subsection{Fines and Suspensions}
Whenever an Archer infringes upon one or more articles from the \Asta , \Ahr\ or \Asr\, or in case of infringing upon regulations based on the rules described in these documents, the board can punish the violator with a fine and/or suspension. The severity of this measure will be determined depending on the specific infringement. In case of disagreement between the board and the violator, the matter will be mediated by a GA.

\section{Committees}
\subsection{Establishment}
The board can establish a committee with a clear task description by communicating its establishment at a GA or by using a written notice, sent to members and recreationists, which includes the task description. This written notice will include the initial committee members.

\subsection{Dissolution}
\subsubsection{Dissolution by the board}
The committee can be dissolved by the board or GA when they have decided that its task is accomplished or obsolete, or if all tasks have been distributed over other committees (or the board). When a committee is manually disestablished, the board will sent a written notice of dissolution to all members and recreationists.

\subsubsection{Dissolution upon task completion}
The committee can also be dissolved automatically, if this was decided at establishment. These committees will be automatically dissolved after they have accomplished their tasks, and are used for one-time events. When such a committee has completed their tasks, a written notice of dissolution will be sent to all members and recreationists by the board.

\subsubsection{Dissolution notice}
Dissolution notices for committees should contain when the committee will be dissolved and a final report about their tasks.

\subsection{Structure}
Every committee has a chairman, who is responsible for presenting results to the board and maintaining order and effective task execution within the committee. Every committee will also be appointed a board member as a board contact, who will serve as a first contact point, and will stay up to date with current committee tasks.

\subsection{Reporting}
Committees are tasked with reporting their status to the board before every GA. Committees also have to report their status to the board when asked. Committees can present regarding their committee at the GA, provided they provide their report to the board at least 14 days before the GA, such that it can be archived.

\subsection{Committee members}
\subsubsection{Joining}
Both Members and Recreationists can join committees after approval of the committee chairman. The board or GA can object to new committee members, provided that they supply both the joining member, as well as the committee itself of a written reasoning. The committee chairman is responsible for informing the board when members has requested to join committees.

\subsubsection{Leaving}
Committee members can leave their committee after they have completed or transferred all their tasks. If tasks cannot be transferred to other committee members, the leaving member will have to complete their task first. The committee chairman is responsible for informing the board when members have requested to leave and can decide if one is allowed to leave the committee. \\ 

The board or GA can also force members to leave committees, provided they supply both the leaving member, as well as the committee itself of a written reasoning.

\subsubsection{Disputes}
In case that disputes arise between the board and committees (and its members), the GA will mediate.

\subsection{Budgets}
Committees have to pass every planned purchase by the Treasurer, who will approve or deny requests. Planned purchases totaling to more than €100,- require a Budget to be submitted to, and approved by the board. Budgets totaling more than €350,- have to be approved by the GA.

\subsection{Workgroups}
Committees may have workgroups, which can have members from outside the committee, to take care of a very specific task. These workgroups have a chairman, who is in the committee as well, and reports to the committee. Budgetting, reporting, joining and leaving go through the committee.

\section{Club Room Rules}
\subsection{Usage rules}
The Club Room may be used by Members and Recreationists at Da Vinci, as well as by participants of the Beginners' Course, as well as those who have been granted access by the board. 

\subsection{Safety}
To prevent holdups at the shooting range, the door to the Club Room will be closed whenever nobody is present, or when archers are present for longer than a minute. The door will only be reopened after knocking, and when the shooting has stopped.

\subsection{Maintenance and Cleaning of the Club Room}
Members and Recreationists can be requested to help with maintenance and cleaning of the Club Room by the board. Maintenance costs will be paid from the association treasury. If necessary, financial help may be asked for from the SSC.

\subsubsection{Small tasks}
Members and Recreationists are expected to perform small maintenance, such as throwing out the trash, themselves without board involvement.

\subsection{Storing personal belongings}
Members and Recreationists are permitted to store their sporting equipment, such as bows and arrows in the Club Room in spots destined for storage, as decided by the board. These materials are stored without any insurance policy, and are stored at own responsibility for the owner of the materials. The board will ensure safety of stored materials on a best effort basis. Stored items must be labeled with some indication of the owners identity. The board may also request the removal of items with a one month notice.

\section{Club Materials}
\subsection{Definition}
Club Materials include bows, arrows, personal protection material, spare parts, (outdoor) targets, target faces, other material owned by the club and are used for archery. \\

Club Materials can be used by any person who is permitted by the board to use the materials. The board can always limit usage of Club Materials if needed. The Club Materials are primarily used during the Beginners' Course and Workshops, other usage is permitted only if this activities are not compromised.

\subsection{Maintenance and Repairs}
\subsubsection{Beginners' Course and Workshops}
During the Beginners' Course and Workshops, trainers are responsible for inspecting the material for damage at the beginning and end of each training or workshop. Damaged materials must not be used until repaired if safety is compromised. The usage of club materials outside the Beginners' Course and Workshops is regulated according to Section \ref{section:clubMaterialInCourse}.

\subsubsection{General}
The board is responsible for periodic and corrective maintenance of the Club Materials to ensure the materials are safe to use. The board can delegate maintenance to the Material Committee. 

\subsubsection{Damages}
In case of any damage during usage in General Training, free practice or external use, the archer who used the bow is responsible for contacting the board. In case the damage occurred to misuse, the archer is also responsible for repair costs. In case of normal wear and tear it is highly appreciated that archers who use the material play an active role in repairing it.

\subsection{Internal Use}
\subsubsection{Beginners}
Beginners are required to use Club Materials during the Beginners' Course.

\subsubsection{General}
After the Beginners' Course, Members and Recreationists are allowed to use the Club Materials during training and free practice. In case there are not enough bow and/or arrows available, the bows are distributed on a first come first serve basis.

\subsubsection{Costs}
During the first 6 months, using club bows and arrows is free of charge. After 6 months a small fee needs to be payed per club evening (now: 0,50 euro). The board and trainers from the SSC can always claim a bow for instruction purposes. In case someone payed a fee for that bow that evening, the fee will be restituted.

\subsection{External Use}
Members are allowed to borrow Club Material for use in external competitions. When borrowing a fee is required equal to the fee of two evenings for each period between two club evenings. If the bow is not returned before the beginning of the next club evening, a fee for that evening needs is required. A deposit of 50 euros needs to be paid, any damage needs to payed from this deposit. In case damage exceeds the amount of the deposit, the member is still required to pay for the damage. \\

In case Club Material is needed for a beginners course, the material needs to be returned in good order at the latest at the start of the lesson. In case the material is not in good order or returned too late a fine of 10 euro is to be paid. If no valid reason is provided for not returning the material in good order on time, the board can decide that the Member is no longer allowed to use Club Materials.

\end{document}